\chapter{Introduction}
%! Introduzione alla ricerca operativa in generale
The subject of \textit{operations research} is defined as "\textit{the application of scientific and especially mathematical methods to the study and analysis of problems involving complex systems}" (see Merriam-Webster definition).
Historically, the origins of operations research can be set during World War II, when the British Army employed scientists as consultants for military strategies based on scientific considerations \cite{ozzola2007}. The success of operations research in the military field led to its application in other contexts after the war, the most important of which was the industrial one. The subject of operations research has benefited immensely from the advent of computers, which allowed to make the large number of calculations required by the most complex problems. \par
Given its obvious mathematical and statistical bases, operations research is not easily accessible for people with a low affinity for scientific subjects. Luckily, in literature there exist works that tackle this problem and aim at letting people know about operations research in a colloquial manner, without the need of a large scientific background, such as the book by V. Ozzola \cite{ozzola2007}, in Italian. \par  
The first step of any operations research application is the formulation of the real problem of interest as a mathematical problem or model, that should capture as much aspects and behaviors of the real one as possible. Then the mathematical problem is solved by means of an algorithm, which can be aimed at finding the optimal solution, or in alternative, one that is reasonably close to it and feasible in practice. The latter types of algorithms are referred to as \textit{heuristics}. \par 
Nowadays, operations research is a pervasive subject that spans through several areas of application, and whose optimization principle can be observed even in nature, e.g. in the behavior of ant colonies.
This thesis, however, focuses on the approximation side of the subject, specifically on the realm of heuristics. \par 
In many applications, nearly optimal solutions are often sufficient, and sometimes the optimal solution is not feasible in practice. Also, exact algorithms that aim at finding the optimal solution are not effective for many pivotal problems, which are too large or too hard. These considerations underline the importance of heuristic algorithms in operations research. \par 
This thesis focuses on the study, implementation and testing of a general purpose heuristic algorithm, with the purpose of proposing an improved version of it. \\ \\ \vspace{0.2cm} 
%! Organizzazione della tesi: presentazione in sequenza di quanto trattat nei singoli capitoli
The thesis is organized as follows. \par
%* CHAPTER 2: Mixed integer programming
\textbf{Chapter~\ref{ch:mips}} introduces the concept of \textit{mixed integer programming}, a process aimed at solving, generally by means of branch-and-cut algorithms, problem instances that are referred to as \textit{mixed integer programs}, or MIPs in short, introduced in Section~\ref{sec:mip}. Numerous software solutions have been developed to generalize and automate the MIP solving process: these are known as MIP \textit{solvers} and are introduced in the second part of the chapter, namely Section~\ref{sec:mipsolvers}. \vspace{0.2cm} \par
%* CHAPTER 3: Primal heuristics
\textbf{Chapter~\ref{ch:primalheur}} dives into the realm of heuristics for MIP problems, in particular that of \textit{primal heuristics}, which are algorithms that aim at finding and improving feasible solutions in the early stages of the branch-and-cut algorithm. The chapter provides an overview of the main primal heuristics, which can be divided into two categories: \textit{start heuristics}, presented in Section~\ref{sec:startheur}, aim at just finding a feasible solution as early as possible in the branch-and-cut search; whereas \textit{improvement heuristics}, presented in Section~\ref{sec:improvementheur}, take the current incumbent solution and from it try to construct a feasible solution with a better objective value. The category of start heuristics can be split into two sub-categories, namely \textit{diving heuristics} and \textit{rounding heuristics}. \vspace{0.2cm} \par 
This thesis focuses on the latter type of primal heuristics, to which the ZI-Round heuristic belongs to. In particular, the focus is put on the implementation and testing of different versions of the ZI-Round heuristic, in an attempt to propose an improved default version with respect to the current one known in literature and originally proposed by C. Wallace in \cite{wallace2010} as an extension of \textit{simple rounding}, a straightforward rounding heuristic. \vspace{0.2cm} \par 
%* CHAPTER 4: ZI-Round MIP rounding heuristic
\textbf{Chapter~\ref{ch:ziround}} provides an in-depth description of the ZI-Round heuristic. In particular, Section~\ref{sec:simplerounding} presents the \textit{simple rounding} heuristic, which is the base from which ZI-Round was developed; while Section~\ref{sec:ziround} presents the actual heuristic, main subject of this thesis. \vspace{0.2cm} \par 
%* CHAPTER 5: ZI-Round extensions
\textbf{Chapter~\ref{ch:ziroundextensions}} presents the versions of ZI-Round implemented and tested. In particular, those are described in Section~\ref{sec:default} to~\ref{sec:proposed}, the last one being the proposed version. \vspace{0.2cm} \par 
%* CHAPTER 6: Computational results
\textbf{Chapter~\ref{ch:compresults}} presents the experimental work involved in this thesis and the computational results obtained. In particular, Section~\ref{sec:preliminary-exp} presents two preliminary experiments, aimed at assessing the reliability of the ZI-Round implementation made by the author of this thesis. Section~\ref{sec:expsetup} describes the organization of the main experiments, involving the hardware and software tools used, the collection of the instances for the test-bed, and the measures used to evaluate the performance and effectiveness of the ZI-Round heuristic on the single problem instances and on the whole test-bed. Section~\ref{sec:expresults} presents and discusses the computational results obtained in the experiments from two standpoints: ($1$) the general performance of the heuristic on the test-bed; ($2$) the behavior of the solution fractionality and cost throughout the execution of the heuristic. \vspace{0.2cm} \par 
%* CHAPTER 7: Conclusions
Finally, \textbf{Chapter~\ref{ch:conclusions}} summarizes the work done in this thesis and draws the final conclusions.