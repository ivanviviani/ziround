\chapter{Conclusions} \label{ch:conclusions}
%! Conclusions
In this thesis the rounding heuristic for MIP problems known as ZI-Round and originally proposed by C. Wallace in \cite{wallace2010} has been analyzed and implemented. Furthermore, some variations of the default version of the heuristic have been described and implemented, including the one proposed. \par
To give an idea of the general context, the thesis first introduced to the reader the concept of mixed integer programming, which involves the solving of mixed integer programs (MIPs), generally by means of sophisticated software products, known as MIP solvers. \par 
The importance of heuristics in operations research, underlined in the introduction, motivated the brief overview about primal heuristics, which are of great importance for finding good incumbent solutions early in the solution process, but can also be used in a standalone manner to find nearly optimal solutions. The classification of primal heuristics led the reader to the specific category of rounding heuristics, to which ZI-Round belongs. \par 
After the theoretical analysis and the implementation of the ZI-Round heuristic and its extensions, the experimental part of this thesis comprised: a couple of preliminary experiments to assess the reliability of the implementation; the organization of the main experiments in terms of software tools, collection of the test-bed, and measures to consider for the performance evaluation; and finally, the presentation and discussion of the computational results obtained in the experiments. In order to limit the effect of performance variability and at the same time artificially increase the size of the test-bed, three different random seeds have been used. \par 
The proposed version of ZI-Round includes: (1) the extension that sorts the singletons of each constraint in ascending order of their objective function coefficients; (2) the extension that resolves fractionality tie-breaks by worsening the objective value; (3) the extension that starts shifting the non-fractional integer variables to improve the objective value only after no more fractional integer variables can be rounded. \par 
The experimental results obtained showed that the proposed version of ZI-Round achieves the greatest success rate among the single extensions and the default version, on both the regular and presolved test-beds, with a higher value on the regular test-bed. As for the average solution gap, ZI-Round seemed to perform better on the regular test-bed, for which the proposed version obtained the lowest gap among the single extensions and the default version. The extension that worsens the objective value when fractionality tie-breaks occur seemed to penalize the proposed version in its average solution gap on the presolved test-bed. Regarding the execution times, the measures confirm the fact that ZI-Round is a low time consuming heuristic with respect to the time spent to solve the LP relaxation. \par 
The results obtained in this thesis showed the improvement of the ZI-Round heuristic, therefore suggesting the use of the proposed version as the new default ZI-Round.